\chapter{Final conclusion} \label{ch:finalconclusion}

%%How is it possible for the UR5 to pick and place the rotors in the desired orientation
%and  location  of  the  work-cell,  while  upholding  the  cycle  limit  of  26  seconds  and
%safety requirements of the work-cell?
Through the use of polar coordinates it was possible to derive, at the x, y coordinates in cartesian space the z had to be assumed, however if the dimensions of the machine were more accurate, it would be possible to have the z axis represented. The orientation of the rotor is given by the layout of the machines. The inverse kinematics is used to calculate the angle for each joint. When the joint angles are known to each location of the process, the trajectories can be computed.\\
The UR5's teach-pendant made it possible for the group to pick and place the rotor in the desired orientations and make a program that allowed the group to place the rotor in the desired cartesian points.\\
The cycle time of the rotor was tested as a half-work-cell so the rotor cycle time would pass the test when it did not exceed 13 seconds. The safety of the work-cell required several emergency stops and the UR5 was also tested for its speed when a person was to enter the work-cell.\\
The conclusion is: The acceptance test was successful and the requirements upheld.