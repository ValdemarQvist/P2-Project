\section{Initial problem statement}\label{ch:Initial problem statment}
%How is it possible to give a robotic arm a set of problems to solve, while figuring out the
%coordinate systems of the 3D-space and giving it a set of code to autonomously figure out the algorithm given to it.\\

%This section will identify the initial problems related to the project case given by Grundfos, which will be discussed and examined for this project.

The following questions will be answered throughout the report, to initiate the research for the work-cell process.

\begin{itemize}
    \item Which ISO standards and regulations does the work-cell require?
    \item How is it possible for the UR to collaborate safely with humans?
%\item Which social aspects would impact the employees when implementing the robot?
%\item How is it possible to optimize the UR within the work-cell?
    \item Is the UR manipulator the most suitable for this task?
%\item What is the most efficient design for the end-effector?
    \item How can the robot localize the objects and place the tool correctly?
\end{itemize}