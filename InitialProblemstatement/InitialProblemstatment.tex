
\newpage
\section{Initial problem statement}\label{ch:Initial problem statment}
%How is it possible to give a robotic arm a set of problems to solve, while figuring out the
%coordinate systems of the 3D-space and giving it a set of code to autonomously figure out the algorithm given to it.\\

%This section will identify the initial problems related to the project case given by Grundfos, which will be discussed and examined for this project.

In this project, a work-cell must be created, which can complete the set task of processing pump rotors through 3 different processes and a visual inspection, within a time limit of 26 seconds. The work-cell must be suitable for the collaboration of human employees and robotic manipulators, and be ergonomic for these employees, in order to create a good and safe work environment. \\
To research this the following questions will be answered throughout the report:

\begin{itemize}

    \item How is it possible to output a balanced, checked, and engraved L40 rotor put in a trolley?
    \item How is it possible to design a work-cell that a manipulator can operate within?
    \item How can each rotor within the work-cell be correctly processed within the required time frame?
    \item How is it possible for the manipulator to collaborate safely with humans?
    \item Which legal standards and regulations are required for the work-cell?
\end{itemize}

%\item Which social aspects would impact the employees when implementing the robot?
%\item How is it possible to optimize the UR within the work-cell?
%\item Is the UR manipulator the most suitable for this task?
%\item What is the most efficient design for the end-effector?
%\item How can the robot localize the objects and place the tool correctly?
