\section{Initial problem statement}\label{ch:Initial problem statment}
%How is it possible to give a robotic arm a set of problems to solve, while figuring out the
%coordinate systems of the 3D-space and giving it a set of code to autonomously figure out the algorithm given to it.\\

%This section will identify the initial problems related to the project case given by Grundfos, which will be discussed and examined for this project.

To initiate the research, the group must find out how a robot can fulfill the work-cell's requirements, while collaborating with humans and reduce the workload of the workers at Grundfos.\\
The following questions will be answered throughout the report:

\begin{itemize}

    \item How is it possible to output a balanced, checked, and engraved L40 rotor put in a trolley?
    \item How is it possible to reduce the running cost, while reducing the work load?
    \item How is possible to design a work-cell that a manipulator can operate within?
    \item How can a rotor be moved within the work-cell in the required time frame?
    \item How is it possible for the manipulator to collaborate safely with humans?
    \item Which standards and regulations does the work-cell require?
\end{itemize}

%\item Which social aspects would impact the employees when implementing the robot?
%\item How is it possible to optimize the UR within the work-cell?
%\item Is the UR manipulator the most suitable for this task?
%\item What is the most efficient design for the end-effector?
%\item How can the robot localize the objects and place the tool correctly?
