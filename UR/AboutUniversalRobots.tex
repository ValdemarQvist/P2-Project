\chapter{UR and LBR iiwa 7 R800} \label{ch:UR}

\section{UR5 Versus LBR iiwa 7 R800}

In this chapter the project takes focus in the differences in these 2 cobots. Are there any which is most suited for the work-process of the work-cell?

\subsection{UR}

Universal Robots was founded in 2005 by a group of Danish engineers. Their thought was, that every industrial robot on the market was designed to be big, heavy and expensive. Therefore the group decided to make a smaller and more agile kind of industrial robot, hence the UR.\cite{Urhist}\\


The UR is now being used to several different repetitive tasks and by big and small companies.
Because UR robots are much smaller, don't have the same force as their bigger counterparts, they can be used in cooperation with humans, which is one of the biggest advantages. This allows the robots to be much more flexible. 


\subsection{UR5 specifications}

These robots can work with humans, and are called collaborative robots/Cobots.\\
Here are the specification for the UR5:\\ 
\begin{itemize}
    \item Weight: 18.4 kg.
    \item Payload: 5 kg.
    \item Footprint: 149 mm.
    \item Joints: +-360 degrees on all the joints.
    \item Operating life: 35,000 Hours.
    \item Speed: joints = 180 degrees/sec, tool = 1 m/sec.
    \item Reach: 850 mm.
    \item Repeatability: +/- 0.1 mm.
\end{itemize}
The materials used on all the cobots are aluminum and plastic\cite{Ur5_about}\cite{UR5_tech}.


\subsection{Kuka}

The Kuka company started for over a hundred years ago. They were the first to invent point welding gripper in Germany.\\
As the years goes by the Kuka company writes history by inventing the world first industrial lightweight robot with sensors in every axis.\\

LBR iiwa and UR are competitors since they have the same market, hence is why this project will take both in to consideration of a possible optimization of the work-cell\cite{KukaHist}.\\


\subsection{LBR iiwa 7 R800}

Here are the specification for the LBR iiwa 7 R800:\\

\begin{itemize}
    \item Weight: 23.9 kg.
    \item Payload: 7 kg.
    \item Footprint: 136 mm.
    \item Joints: Ranging from +-120 degrees to +-170.
    \item Operating life: 30,000 Hours.
    \item Speed: Joints = 180 degrees/sec
    \item Reach: 800-820 mm
    \item Repeatability: +/- 0.1 mm.
\end{itemize}
\cite{KukaSpec1},\cite{KukaSpec2}.

\section{Which Cobot is more suitable?}

Given the case-description \ref{ch:case description}, the cobot needs to complete the cyclus of the work-cell within 26 seconds. Based on the work-cell, the range of the 