\chapter{Final conclusion} \label{ch:finalconclusion}

%%How is it possible for the UR5 to pick and place the rotors in the desired orientation
%and  location  of  the  work-cell,  while  upholding  the  cycle  limit  of  26  seconds  and
%safety requirements of the work-cell?
%Through the use of polar coordinates it was possible to derive, at the x, y coordinates in cartesian space the z had to be assumed, however if the dimensions of the machine were more accurate, it would be possible to have the z axis represented. The orientation of the rotor is given by the layout of the machines. The inverse kinematics is used to calculate the angle for each joint. When the joint angles are known to each location of the process, the trajectories can be computed.\\
%The UR5's teach-pendant made it possible for the group to pick and place the rotor in the desired orientations and make a program that allowed the group to place the rotor in the desired cartesian points.\\
%The cycle time of the rotor was tested as a half-work-cell so the rotor cycle time would pass the test when it did not exceed 13 seconds. The safety of the work-cell required several emergency stops and the UR5 was also tested for its speed when a person was to enter the work-cell.\\
%The conclusion is: The acceptance test was successful and the requirements upheld.

Through a contextual and technical analysis, the group could commence the design phase and the  requirements of the work-cell was found. The requirements were simple yet precise for the assignment to succeed. Through the requirements an acceptance test was formed.\\
The technical analysis was based on an analysis of how the work-cell and manipulator could work together, therefore the group had to analyse all the different components of the work-cell while keeping the pick and place assignment in mind. When the data of the technical analysis was on place the group moved to the design phase where a desired work-cell was implemented, so the idea of using all the different sensors and placement of the manipulators could be visualized. After the design a crucial decision was made to make 2 UR5's work together to up the speed of the pick and place process. Therefore trajectories of the UR5 had to be made to get to know how the robot would calculate itself through a 3D space with obstacles. This was then taken in to consideration and used in an example from the groups section \ref{ch:kinematics}, and in the attached maple document.\\
When considering the whole work-cell safety measures had to be tested and analyzed as well. Safety requirements is of utmost importance since the work-cell will need workers inside it from time to time. Cobots are normally not dangerous but with high speed of the end effector and a mildly sharp iron product in it, it can be. The group then took emergency stops and end effector speed in to consideration and the tests could be done.\\
When the testing commenced the requirements was upheld and the solution for the assignment was successful. 