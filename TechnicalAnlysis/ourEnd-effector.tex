\section{End-effector}

In this section the 3 most used gripper types being used in the industry, will be explained.

\subsection{Mechanical gripper}
The End-effector which is available for the group in this project is a pneumatic driven gripper. The gripper weighs 0.5 Kg (excluding gripper fingers). It has an operation pressure range from 3 to 7 Bar and the gripper claws are operated independently\cite{MWC}. The gripper in its neutral state is open, when air pressure is applied to the gripper it will close.\\
To control the gripper, solenoid actuated valves are used. To operate the valve a voltage is applied to the solenoid and the valve will move to an open position and let air flow towards the gripper.\\
The group can use the gripper by connecting the valve to the control box of the manipulator, then control the valve through the teach pendant.\\ 

\begin{figure}[H]
    \centering
    \includegraphics{TechnicalAnlysis/MWforcedia.PNG}
    \caption{This diagram shows the force on the object, depending on the pressure and length of the gripper fingers\cite{MWC}}
    \label{fig:Mvforce}
\end{figure}

\subsection{Suction / vacuum cup}
This kind of gripper is mostly used for pick and placement of non-ferrous materials, where only one surface is available to handle the material.\\ 
It functions by causing a vacuum inside the cups to hold onto.\\
This type of gripper is good for picking up objects, which are smooth, flat, and clean, and is less suitable for picking up objects with holes in it \cite{Gripper1}.\\

\subsection{Magnetic gripper}
The magnetized end-effector is functioning much like the vacuum cups, where it only can hold on to a single surface on ferrous materials. Instead of functioning by vacuum, it is using magnets to hold onto materials. The gripper is switching between on/off mode by pushing compressed air into the case, to push the magnet on or off see fig \cite{MagnetCite}.\\
This type of end-effector has the drawback that, if fast movements occur, there is a chance for the material to slip off its grip. This can also occur for the vacuum gripper\cite{Gripper2}.\\

\begin{figure}[H]
    \centering
    \includegraphics{TechnicalAnlysis/MagnetGripper.PNG}
    \caption{This figure shows how a basic Magnetic Gripper is turned on/off \cite{MagnetCite}}
    \label{fig:MagnetGrip}
\end{figure}

\subsection{Inflatable Gripper}
Inflatable bladder type grippers, function like a balloon, which will be expanded once it is in the holding grip of the material, and is used to move items which requires careful handling \cite{Gripper3}.\\

\section{Repeatability and accuracy}
Repeatability should not be confused with accuracy, though they are somewhat similar.\\
Repeatability is the factor of how high the spread rate is, which means a higher repeatability will cause the end-effector to reach closer to the same point on every repeat action.\\
Accuracy is the factor which tells us how close the end-effector can be expected to move to a given position set by the user. This means a higher accuracy will lead to a diminished spread, compared to the position given in a space\cite{RepeatWhat}.\\

\begin{figure}[H]
    \centering
    \includegraphics[width=.9\textwidth]{TechnicalAnlysis/RepeatTest.jpg}
    \caption{The difference between high/low repeatability and accuracy\cite{RepeatWhat}}
    \label{fig:Repeatability}
\end{figure}
