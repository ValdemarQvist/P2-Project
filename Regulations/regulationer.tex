\section{Safety and Regulations}\label{ch:regulation}
To secure the work-space environment, the company has to take in to consideration how to limit the big accidents. This is done by having a set of rules to follow.\\

\subsection{APV}
APV is a workplace-assessment that takes risks into consideration. APV's can differ in many ways, depending on what type of workplace it is. The danish work-supervision branch has made checklists to ease the risk evaluations of work-spaces. \cite{Apv}\cite{Risikovurdering}.\\


\subsection{ATEX}
ATEX is the regulations for an area where there can be explosion danger. Areas where there is a need for ATEX can be a dust, chemicals or specific types of gasses. The workplace will need to be mapped for different risks of explosion danger, or what can be done to minimize the risk.\\
At Grundfos, they have designated areas where chemicals are stored, the main explosion danger is the dust which settles in the work area, this can be overcome by regularly cleaning the area\cite{ATEX}.  

\subsection{ISO 10218-1:2011(E)Robots and robotic devices Part 1: Robots}
This standard 10218-1 is covering the safety aspects for the design and constructing of a robot.
The following will apply for every industrial robot for the international standards:\cite{Robotterdel1ds}.\\
This part does not consider the noise level of the robot.
\begin{itemize}
    \item A robot needs to operate in accordance with its program.
    \item The robot needs to have it's task stated before executing its program.
    \item State the defined work-space for which a purposely designed robot will cooperate with humans.
    \item The collaborative work-space needs to be safeguarded while humans and robots simultaneously can perform tasks.
    \item Dangerous movements by the robot, inside the work-space is not allowed.
    \item It's specifically instructed to place the end-effector on the mechanical surface, so the robot can do the tasks.
    \item The robot needs to have a protective stop.
    \item The robot needs to have safety rated speed and reduced speed.
\end{itemize}

\section{ISO 10218-2:2011(E)Robots and robotic devices Part 2: Robot systems and integration Safety requirements}\label{ISO2}
This standard 10218-2 covers guidelines for protecting and safeguarding personal, for implementation and testing when the robot is in operation, if robot needs maintaining or repair.

\subsection{safety requirements EN ISO 10218-2:2011}

\S5.3.8.1 Every robot cell or system is required to have at least one independent emergency stop function, each panel that is capable of moving the robot must have access to a manual emergency stop, that complies with the requirements of ISO 13850 and IEC 60204-1.\\ 
\\
\S6.9.2 These robotic systems must have a single emergency stop function, that affect any relevant parts of the robot.\\
\\
\S5.3.9 The robot system must be installed in such way, that emergency or protective stops does not result in a hazard or dangerous situation for any personal.\\
\\
\S5.7.4 When working with collaborative robots, hand guiding may be used for the collaborative part of the task. When this method is used, it must meet the requirements described ISO 10218-1.\\
\\
\S5.8 The robotic system must include a procedure for periodic testing of any safety-related equipment. The information of use for the robot must include a description of the testing procedure. \\

\subsection{Collaborative robot operation \S5.11}

\S5.11.1 Collaborative robot operations are only used for predetermined tasks, which is only possible when all the required safety measures are present. These operations are only for robots specifically designed for collaborative tasks complying with ISO 10218-1. \\
\\
\S5.11.2 Due to the potential of physical contact between the personal and the robot, protective measures must be provided to ensure the operators safety. This includes; \\
A: A risk assessment as described in \S4.3\\
B: Robots used in the work-space must meet the requirements of ISO 10218-1 \\
\cite{ISO10218-2}


