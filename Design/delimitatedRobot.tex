\chapter{UR5 - Delimited Cobot}

In this chapter the specifications of the delimited cobot will be presented.

\section{UR}

Universal Robots was founded in 2005 by a group of Danish engineers. Their thought was, that every industrial robot on the market was designed to be big, heavy and expensive. Therefore the group decided to make a smaller and more agile kind of industrial robot.\cite{Urhist}\\
Here are the specification for the UR5:\\ 

\subsubsection{UR5}

\begin{itemize}
    \item Weight: 18.4 kg.
    \item Payload: 5 kg.
    \item Footprint: 149 mm.
    \item Joints: +/-360 degrees on all the joints.
    \item Operating life: 35,000 Hours.
    \item Speed: joints = 180 degrees/sec, tool = 1 m/sec.
    \item Reach: 850 mm.
    \item Repeatability: +/- 0.1 mm.
\end{itemize}

The materials used on all the cobots are aluminum and plastic\cite{Ur5_about}\cite{UR5_tech}.\\

\begin{figure}
    \centering
    \includegraphics[width=9cm]{UR/UR5pic.jpg}
    \caption{Universal Robots UR5 \cite{UR5billede}}
    \label{fig:UR5}
\end{figure}

Looking in to the specifications of the UR5 some delimited requirements can be written:\\
 



\section{Delimited requirement specifications}

\begin{itemize}
    \item A rotor cycle may not exceed 26 seconds within the work-cell
    \item The cycle of the rotor inside the work-cell must be ergonomic.
    \item The emergency stop must always be in reach of an operator.
    \item The UR5 must operate the rotor faster than a human.
    \item At least 4 points (Machines) in the work-cell must be within reach of the UR5.
    \item The UR5 must be able to lift a payload of 645g in 0 posistion.
    \item The UR5's programming must be flexible, so it wont lower the production time when a new machine is presented.
    \item The UR5 must react to the different tasks given to it by sensors.
\end{itemize}