
\chapter{Acceptance test}\label{ch:Test}

In this phase of the project, a series of tests will be performed, in order to find out if the solution can fulfill the set requirements.\\ 

\section{Test 1 - Cycle time}

\paragraph{Testing Requirement 1:} The cycle time of the rotor must not exceed 26 seconds.\\
In order to test this requirement, the cycle time of the first half of the entire process is tested with the UR5. The cycle in this test therefore consists of moving the rotor from the conveyor belt to the balancing machine, then on to the visual inspection, and at last place it on the queue table. Since this is only half the steps of the entire process see \ref{fig:first-part}, this cycle is expected to be performed in 13 seconds or less. 

\subsubsection{Setup of the test}

\begin{itemize}
   \item A test rotor.
   \item A wooden frame simulating the drawer in the balancing machine.
   \item A fixture to simulate a hole in the queue table for the rotor to be placed in.
   
\end{itemize}

\paragraph{Expected outcome:}
The UR5 moves the rotor to each required point in 13 seconds or less. 

\paragraph{YouTube video Link:}

\paragraph{Results:}
The UR5 moved the rotor from the conveyor belt to each required position in approximately 13 seconds. 

\section{Test 2 - Ergonomics and signals.}

In this test the group will test the UR5 ability to react upon signals from the visual inspection, while keeping the work-flow intact.\\
Testing Requirements, see \ref{ch:Delimitations}:

\paragraph{Requirement 2:} It is required that the cobot differentiates the height of the rotor for visual inspection every hour. so the visual inspector is guided up from the static position throughout the entire workday.
\paragraph{Testing Requirement 6:} The UR5 has to perform different task depending on signals it is receiving.
\subsubsection{Setup of the test}

The group made the UR5 move by coding an if-else loop. The main loop consisted of differentiating places of the rotor to be inspected, respectively 12 and 42 cm's above the table, so that the worker would switch between sitting and standing up depending on the height the rotor is being held.\\
When the button was pressed, this would take the UR5 into a different trajectory, which would mimic when a rotor failed the inspection.

\begin{itemize}
    \item 1  rotor.
    \item A button to emulate a signal from a machine.
    \item 2 objects at different heights. 
\end{itemize}

\paragraph{Expected outcome:} 
The UR5 is expected to lift the rotor from a height of 12 to 42 cm over the visual inspection table every 10. time, while a inspector will press a signal to move the rotor to the trash pile. 

\paragraph{YouTube video Link:}

\paragraph{Results: }

The worker was guided up in a position where he couldn't sit down, and then to a position where he could rest.\\
When the rotor failed, a new trajectory was taken by the UR5, and the rotor was then "failed". The UR5 would quickly find its starting point again and begin the sequence all over.\\

\section{Test 3 - Emergency stop}

When an emergency stop is pressed the UR5 should stop entirely.\\
Testing Requirements, see  \ref{ch:Delimitations}:\\

\paragraph{Requirement 3:} The emergency stops has to stop the entire work-cell.\\

\subsubsection{Setup of the test}

\begin{itemize}
    \item UR 5.
    \item The UR 5 teach pendant.
    \item 1 emergency stop.
\end{itemize}

\paragraph{Expected outcome:}
The manipulator will stop completely when pressing the emergency-stop.

\paragraph{YouTube video Link:}

\paragraph{Results: }
When the emergency stop on the teach pendant was activated, the manipulator stopped its trajectory. Before the manipulator could move again it was required to reset the emergency stop and teach pendant needed to be pushed.\\

\section{Test 4 - Multiple emergency stops}
The work-cell is a potentially dangerous environment to anyone entering it. The robots move around sharp rotors that could potentially cause injuries, despite the earlier mentioned safety measures. For this reason, it has been decided that an emergency stop button will be placed within reach of the employee performing the visual inspection. Additionally, 7 emergency stop buttons are strategically placed inside the work-cell, so that the distance from the nearest emergency stop button is maximum 55 cm from anywhere inside the work-cell. When an emergency stop is pressed, the UR5 will stop entirely. With this setup, anyone inside the work-cell would be able to immediately stop the production, if any emergency situation should occur. \\
No further testing is needed to fulfill this requirement, since test 3 demonstrates the functionality of the emergency stop buttons. \\

\begin{figure}[H]
    \centering
    \includegraphics[width=0.7\textwidth]{Design/Work_cell_10.png.png}
    \caption{Illustration for the fourth test.}
    \label{fig:fourthtest}
\end{figure}


\section{Test 5 - Payload}
The purpose of this test is to verify that the UR5 is able to carry the required payload of 1145 grams. In order to test this, the UR5 must pick up and hold an object of the required weight in its most stretched out position.

\paragraph{Testing requirement 5:} The UR5 must be able to have a lifting capacity more than the total combined payload of the rotor(645g) and the end-effector(500g) at maximum reach.\\

\subsection{Setup of the test}
The test will be performed by making the UR5 pick up a block of iron(1530.9g) with the end-effector(500g). This will give a combined payload of 2030.9g. The UR5 will then move outward until it is holding the block at its maximum reach. 

 \begin{itemize}
     \item UR5
     \item Object weighing at least 1530.9g.
     \item End-effector 500g.
 \end{itemize}
 
 \paragraph{Expected outcome:}
The UR5 will pick up the block of iron and move it outward until it is at its maximum reach. 

\paragraph{YouTube video Link:}\\
 
\paragraph{Results: }
The UR5 was successful in performing the test. It is therefore concluded that the UR5 can handle the required payload.\\ 

\section{Test 6 - End-effector velocity}
When a worker enters the work-cell, the UR5's End-effector must not exceed 100mm/sec.

Testing Requirements, see  \ref{ch:Delimitations}:

\paragraph{Requirement 7:} When a worker enters the work-cell, the velocity of the end-effector on the UR5 has to slow down to 0,1 m/s, to reduce the risk of harmful behaviour.\\

\subsubsection{Setup of the test}
The velocity of the end-effector at the  300mm/s setting and 100 mm/s setting will be measured by tracking the motion off the end-effector, by recording the motion, with the use of a tracking software.
\begin{itemize}
    \item Measurement paper.
    \item 1 button.
    \item 1 camera
    \item Tracker software.
\end{itemize}

\paragraph{Expected outcome:}
The manipulator will slow down from 300 mm/s to 100 mm/s when the button is pressed.

\paragraph{YouTube video Link:}

\paragraph{Results: }
shows that the velocity at 300 mm/s had a maximum velocity at 350mm/s. The 100m/s had a maximum velocity of 130mm/s, see figure \ref{fig:velocitytest}.\\
This did not meet the expectations of the group, however the noise graph comes to the lack of having a good enough camera to produce a clear picture of the motion and may have yielded a better result. To meet the requirement of a maximum velocity of 100 mm/s, a lower velocity than 100mm/s will be used to make sure that the end-effector do not surpass the requirement.\\ 

\begin{figure}[H]
  \centering
    \includegraphics[width=\textwidth]{Design/velcit.PNG}
    \caption{The velocity of the end-effector at 300mm/s setting on the left and on the right 100mm/s setting}  \label{fig:velocitytest}
\end{figure}

\section{Test 7 -Accuracy and repeatability}
In this test, item number 8 from the delimitation will be tested. It is important in this pick and place scenario to place the rotor in the correct place every time. 

\subsubsection{Setup of the test}

\begin{itemize}
 \item 2 pieces of paper with circles 
 \item A pen 
\end{itemize}

The UR 5 will be programmed in a loop to make a dot ten times within the circles on each paper. After that it will be measured to make sure that it is within the requirements.

\paragraph{Expected outcome:}
Due to the specifications on the UR5, which has a repeatable accuracy of 0.1mm. The exceptions for the this test is that any of the dots is inside a radius of 0.1mm.
\paragraph{YouTube video Link:}

\paragraph{Results: }
Shows that the UR5 could perform the task of being accurate within 0.1mm and upheld the requirements of being within the radius of 1mm. 



%\section{Test 5: }
\section{Test discussions}



 

 