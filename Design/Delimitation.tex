\section{Delimitation's} \label{ch:Delimitations}

The requirements for the desired robot is not what is available for the project. 
The work-cell that is given to the group by AAU, does not have enough $m^2$ for the manipulator to use it maximum reach in all directions due to a safety fence around the work-cell. As the case describes, see \ref{fig:workcellMR}, the reach of the manipulator has to be 2 meters. The ideal work-cell can simply not be cut down to a shorter distance, without having to close the work-cell completely from entry.\\
Delimiting the reach, a new solution had to be made, which is why the group decided to implement another manipulator on each side of the pallet-table. This has to be delimited as well since the group doesn't have permission or knowledge to implement 2 UR5's on the same table.
For testing purposes, the group will focus its tests one half of the work-cell. \\
The required speed for the manipulator can in theory be done since the manipulator assigned to the group has a speed of 1m/s. The only delimitation is that, if you want to get this speed, a default setting on the UR5 has to be overridden, so that it will move with maximum speed \cite{UserManual}.\\
Also the machinery have some time-consuming processes, hereby the threshold of 26 seconds might be exceeded.\\
Looking at the sensors, their  is one only available sensor for the group is a door switch which is mounted on the door to the safety fence around the work-cell. This is a delimitation for the safety process of this project. However the sensors can be simulated with another method, for example a manual switch,which will be hard wirded into the control box of the UR5.\\
