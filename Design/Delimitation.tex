\chapter{Delimitations} \label{ch:Delimitations}

The requirements for the ideal robot is not what is available for the project. 
The work-cell that is given to the group by AAU, is not nearly big enough to cover the desired reach of the manipulator. As the case describes, see \ref{fig:workcellMR}, the reach of the manipulator has to be the same as 2 meters. The ideal work-cell can simply not be cut down to a shorter distance, without having to close the work-cell completely from entry.\\
Delimiting the reach, a new solution had to be made, which is why the group decided to implement another manipulator on each side of the pallet-table. This has to be delimited as well since the group doesn't have permission or knowledge to implement 2 UR5's on the same table.\\
The required speed for the manipulator can in theory be done since the manipulator assigned to the group has a speed of 1m/s. Though this has to be delimited as well, since the cooperation of the 2 UR5's is not an option, and without 2 UR5's, the distance of the machines will not be reachable.\\
Also the machinery have some time-consuming processes, hereby the threshold of 26 seconds might be exceeded.\\
Looking at the sensors, the only available sensor for the group is in the UR5, which is the pressure sensor, so the sensors needs to be delimited down to only this one sensor. This is a delimitation for the safety process of this project, since the group can not test the process of slowing the robot down when entering the work-cell.\\
