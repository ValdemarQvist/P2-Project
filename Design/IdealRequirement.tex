\chapter{Requirements specification} \label{ch:IdealReq}

%\begin{itemize}
%    \item A rotor cycle time may not exceed, 26 second within the work-cell.
%    \item The cycle within the work-cell must be ergonomic for the co-worker and reduce work-stress.
%    \item An emergency stop must always be in reach of an operator
%    \item The safety of the work-cell must comply with the ISO standards 10218-2:2011.
%    \item The manipulator must provide a better work-flow than a human alone.
%    \item Each machine must be within reach of at least one of the robotic manipulators.
%    \item The manipulator must be able to lift a payload of 645g at maximum reach.
%    \item The manipulator must be signaled by the sensors in the machines, to handle different tasks.
%    \item The setup must be flexible to accommodate for new machines within.
%\end{itemize}
\section{Delimitations} \label{ch:Delimitations}

The requirements for the ideal robot is not what is available for the project. 
The work-cell that is given to the group by AAU, is not nearly big enough to cover the desired reach of the manipulator. As the case describes, see \ref{fig:workcellMR}, the reach of the manipulator has to be the same as 2 meters. The ideal work-cell can simply not be cut down to a shorter distance, without having to close the work-cell completely from entry.\\
Delimiting the reach, a new solution had to be made, which is why the group decided to implement another manipulator on each side of the pallet-table. This has to be delimited as well since the group doesn't have permission or knowledge to implement 2 UR5's on the same table.
For testing purposes, the work-cell can be simulated in a 2-1 scale, requiring half the space, and making the tests possible. \\
The required speed for the manipulator can in theory be done since the manipulator assigned to the group has a speed of 1m/s. The only delimitation is that, if you want to get this speed, a default setting on the UR5 has to be overridden, so that it will move with maximum speed \cite{UserManual}.\\
Also the machinery have some time-consuming processes, hereby the threshold of 26 seconds might be exceeded.\\
Looking at the sensors, the only available sensor for the group is in the UR5, which is the pressure sensor. This is a delimitation for the safety process of this project. However the sensors can be simulated with another method, for example a command that can be implemented in the code of the UR5.\\

\section{Delimited requirement specifications}

\begin{itemize}
    \item A rotor cycle may not exceed 26 seconds within the work-cell
    \item The cycle of the rotor inside the work-cell must be ergonomic.
    \item The emergency stop must always be in reach of an operator.
    \item The UR5 must operate the rotor faster than a human.
    \item At least 4 points (Machines) in the work-cell must be within reach of the UR5.
    \item The UR5 must be able to lift a payload of 645g in 0 posistion.
    \item The UR5's programming must be flexible, so it wont lower the production time when a new machine is presented.
    \item The UR5 must react to the different tasks given to it by sensors.
\end{itemize}