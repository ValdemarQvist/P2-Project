Industrial robots might use extra safety precautions, to decrease the probability of incidents happening. This could be extra sensors detecting when humans enter, or leave their designated workspace. For caged robots, the door to the cage must usually be closed in order for the robot to run, but you might also want other more reactive ways of ensuring the safety of the personal, particularly when using collaborative robots. One of the most used sensors on these kind of robots would be a collision detection sensor. This allows the robot to register when colliding with a soft surface, and the ability to then either stop, or reduce speed.\\ 

Many other types of safety sensors might be needed when humans are working side by side with robots. This could include cameras, lasers or pressure sensors. Anything that can help letting the robots know, that humans are present.\\ 

When working in a production, that require the robots to pick up parts, and place them elsewhere, it might be a very good idea to have a part detection sensor. This sensor tells the robot whether it picked something up or not, and potentially if it is picked up in the right way. If something goes wrong, the robot might either send an error message to an operator, or try to repeat the task, depending on the configuration.\\
\cite{sensors}