\pdfbookmark[0]{English title page}{label:titlepage_en}
\aautitlepage{%
  \englishprojectinfo{
     Pick and Place%title
  }{%
    Robotics
  }{%
    Spring Semester 2018 %project period
  }{%
    B229 % project group
  }{%
    %list of group members
    Valdemar Jul Qvist\\
    Mark Richard Blankensteiner\\
    Jonathan Midtgaard Jensen\\
    Anders Fischer Steen Jensen\\
    
  }{%
    %list of supervisors
   Jesper Abilgaard Larsen\\
   Lykke Brogaard Bertel 
  }{%
    1 % number of printed copies
  }{%
    \today % date of completion
  }%
}{%department and address
  \textbf{Robotics}\\
  Aalborg University\\
  \href{http://www.aau.dk}{http://www.aau.dk}
}{% the abstract
This report aims to analyze the link between a manipulator and the required setup for the work-cell.\\
The case-description informs that the cycle time for a rotor needs to be 23 seconds or less. The work-task is hereby to communicate with the manipulator and figure out its trajectories to make a smooth and fast process.\\
To figure out which way the project needs to head a contextual analysis must be made and some background history of the company Grundfos is required, to make the ideal idea for the optimization of the rotor.\\
In between these tasks many sub-tasks must be considered such as safety, sensors, kinematics, the setup of the work-cell and how it has to be optimized for a reduced cycle-time. \\
Taking the above into consideration a set of requirements is made and the acceptance tests and implementation of the requirements can commence.
}
